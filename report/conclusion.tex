\section{Conclusion}\label{sec:conclusion}

The OBKF provides the optimal estimation relative to the posterior distribution. The algorithm is obtained just by replacing the covariance matrixes in the classic KF to the posterior ones. The posterior matrixes are estimated by MCMC, and the likelihood function is computed by the sum-prodduct algorithm. 

Since it utilizes the prior knowledge, it doesn't require a lot of data compared with the adaptive filter. For the tracking problem, the OBKF achieves the better estimation than the adaptive method, even though the adaptive method uses much more data.

However, the OBKF is computationally expensive. This problem can be solved in some situation, such as when the environment is static or when the change of the unknown parameter is small. Developing more efficient algorithm is a future work.

Also, the performance of the OBKF is sensitive to the prior distribution. If the prior distribution of the OBKF is very different from the true prior distribution, the OBKF doesn't converge to the optimal estimation.
