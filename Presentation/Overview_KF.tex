\section{Overview Kalman Filter} 
\begin{frame}
    \tableofcontents[currentsection]
\end{frame}

\subsection{Basic Idea}
\begin{frame}{Kalman Filter: Basic idea}
    \pnote{* カルマンフィルタはinputに一つ前の推定をとり、それを観測とcombineしてbetterなものを出力することについて述べる}
\begin{itemize}
    \item Outputs the optimal estimation of the state of a dynamic system
    \item Exploits both the prediction based on the model and the measurement
    \item Robot Localization
\end{itemize}
\end{frame}

\subsection{Algorithm}
\begin{frame}[allowframebreaks]{Kalman Filter: Algorithm}  % ここに引用元が必要
    \pnote{* カルマンフィルタが動作する線形動的システムについて、その表記をまとめているだけである旨を伝える\\
    * このシステムのパラメータがわかっている限り、カルマンフィルタは動く}

Works for linear dynamic systems:
\begin{block}{Linear Dynamic System Example}
\begin{align*}
{\bf x}_{k+1} &= {\bf \Phi}_k{\bf x}_{k} + {\bf \Gamma}_k{\bf u}_{k}\notag\\
{\bf y}_{k} &= {\bf H}_k{\bf x}_{k} + {\bf v}_{k}
\end{align*}   
\begin{itemize}
    \item ${\bf x}_k$: {\it state vector}
    \item ${\bf y}_k$: {\it observation vector}
    \item ${\bf u}_k,\; {\bf v}_k$: {\it zero-mean noise vector}
    \item ${\bf \Phi}_k, \; {\bf \Gamma}_k, \; {\bf H}_k$: {\it transition matrix}
\end{itemize}
\end{block}

\pause

\begin{algorithm}[H]
\caption{Classic Kalman Filter}
    \pnote{* 推定した平均と分散、観測データを利用して、k+1のestimationを出していることについて言う}
\begin{algorithmic}[1]
\REQUIRE ${\bf \hat x}_k$, ${\bf P^x}_{k}$, ${\bf y}_k$
\STATE $\tilde {\bf z}_k = {\bf y}_k - {\bf H}_k{\bf \hat x}_k$
\STATE ${\bf K}_k = {\bf P^x}_k{\bf H}^T_k({\bf H}_k{\bf P^x}_k{\bf H}^T_k+{\bf R})^{-1}$
\STATE $\hat{\bf x}_{k+1} = {\bf \Phi}_k{\bf \hat x}_k+{\bf \Phi}_k{\bf K}_k{\bf \tilde z}_k$
\STATE ${\bf P^x}_{k+1} = {\bf \Phi}_k({\bf I}-{\bf K}_k{\bf H}_k){\bf P^x}_k{\bf \Phi}^T_k + {\bf \Gamma}_k{\bf Q}{\bf \Gamma}_k^T$
\ENSURE ${\bf \hat x}_{k+1}$, ${\bf P^x}_{k+1}$
\end{algorithmic}
\end{algorithm}

\begin{itemize}
    \item ${\bf \hat x}_k$: {\it estimated mean}
    \item ${\bf P^x}_k$: {\it estimated covariance}
    \item ${\bf R}$: {\it Noise covariance of ${\bf u}_k$}
    \item ${\bf Q}$: {\it Noise covariance of ${\bf v}_k$}
\end{itemize}
\end{frame}

\subsection{Problem of Kalman Filter}

\begin{frame}{Problems of Kalman Filter}
    \pnote{* 先ほど見せたR, Qの値が真の値と離れてしまっている場合、Kalman Filterによる推定は精度が悪くなってしまうことについて述べる\\
    * 理解を深めるため、以降で随時r=1である場合について例を出すことにいついて述べる}
\begin{itemize}
    \item The performance is sensitive to the accuracy of the noise covariance matrixes: ${\bf Q}$ and ${\bf R}$  \cite{Sangsuk-Iam1990}
\end{itemize}
\end{frame}

\begin{frame}{Uncertainty of the noise covariances}

Assume the covariances are parameterized by ${\bf \theta}=[\theta_1, \theta_2]$

\begin{itemize}
    \item The covariance matrix of ${\bf u}$ is  ${\bf Q}^{\theta_1}$ (e.g. ${\bf Q}^{\theta_1}$ = $\theta_1{\bf I}$)
    \item The covariance matrix of ${\bf v}$ is ${\bf R}^{\theta_2}$ (e.g. ${\bf R}^{\theta_2}=\theta_2{\bf I}$)
    \item Unknown prameter: ${\bf \theta}=[\theta_1, \theta_2]$
\end{itemize}

If the $\theta$ used in the algorithm is very different from the true $\theta$, Kalman Filter provides a poor estimation
\end{frame}

\subsection{Solutions of the Problem}

\begin{frame}{Solutions of the Problem}
\begin{itemize}
    \item Non-Bayesian Approach: Adaptive Kalman Filter
    \begin{itemize}
        \item Doesn't require any prior knowledge ($\theta$ is not a R.V.)
        \item Requires a lot measured data
    \end{itemize}
    \pause
    \item Bayesian Approach: IBR Kalman Filter
    \begin{itemize}
        \item Require prior knowledge ($\theta$ is a R.V.)
        \item Robust: Guarantees the best average performance relative to the prior distribution
    \end{itemize}
    \pause
    \item Bayesian Approach: Optimal Bayesian Kalman Filter
    \begin{itemize}
        \item Requires prior knowledge ($\theta$ is a R.V.)
        \item Utilizes measured data to estimate unknown parameters
        \item Optimal over the posterior distribution obtained from measured data
    \end{itemize}
\end{itemize}    
\end{frame}